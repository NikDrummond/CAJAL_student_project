\documentclass{article}
\usepackage[margin=0.7in]{geometry}
\usepackage[parfill]{parskip}
\usepackage[utf8]{inputenc}
\usepackage{amsmath,amssymb,amsfonts,amsthm,bm}
\usepackage{graphicx}
\usepackage{titlesec}
\usepackage[table]{xcolor}
\usepackage[colorlinks=true, linkcolor=dblue, citecolor=dred]{hyperref} 
\usepackage[nameinlink]{cleveref}
\usepackage{natbib}
\usepackage{braket}
\usepackage{url}
\usepackage{lmodern}

\definecolor{dred}{rgb}{0.6,0,0}
\definecolor{dpurple}{HTML}{A020F0}
\definecolor{dblue}{rgb}{0,0,0.6}
\Crefname{equation}{Equation}{Equations}
\Crefname{figure}{Figure}{Figures}
\creflabelformat{equation}{#2#1#3}
\crefrangelabelformat{equation}{#3#1#4-#5#2#6}

\renewcommand\b\bm

\begin{document}

\begin{equation}
    C_{ii} = var(m_i) = f (1-f)
\end{equation}

\begin{align}
    dim &= \frac{(\sum_i \lambda_i)^2}{\sum_i \lambda_i^2}\\
    &= \frac{tr(\b{C})^2}{tr(\b{C} \b{C})}\\
    &= \frac{tr(\b{C})^2}{\sum_{ij}C_{ij}^2}\\
    &= \frac{(M f (1-f))^2}{M f^2 (1-f)^2 + \sum_{i \neq j} C_{ij}^2}\\
    &\approx \frac{M^2 f^2 (1-f)^2}{M f^2 (1-f)^2 + M (M-1) \braket{C_{i \neq j}^2}}\\
    &= \frac{M}{1 + (M-1)f^{-2}(1-f)^{-2}\braket{C_{i \neq j}^2}}.
\end{align}

Define $\b{\Sigma} = \b{J}^T \b{J}$ and $\b{z}_{ij} = [z_i; z_j] \sim \mathcal{N}(\b{z}, \b{\mu} = 0, \b{\Sigma}[i;j, i;j] ) $.
\begin{align}
    C_{ij} &= \braket{m_i m_j} - f^2\\
    &= p(z_i > \theta_i \cup  z_j > \theta_j) - f^2\\
    &= \int_{z_i = \theta_i} \left [ \int_{z_j = \theta_j} p(z_j| z_i) dz_j \right ] p(z_i) dz_i.
\end{align}

Define $S_{ij} = s_{ij}^2 = \braket{(\b{j}_i^T \b{x})^T (\b{j}_j^T \b{x})}$ as the output correlation before the nonlinearity.
Can we make a convexity argument about $\braket{C_{ij}^2}$ and $\braket{S_{ij}^2}$ and apply Jensen's inequality?
This would be useful since we can analytically compute $\braket{S_{ij}^2}$:
\begin{align}
    \braket{S_{ij}^2} &= \int S^2 p(\rho|\b{j}_1, \b{j}_2) p(\b{j}_1) p(\b{j}_2) dS d\b{j}_1 d\b{j}_2\\
    &= \int (\b{j}_1^T \b{j}_2)^2 p(\b{j}_1) p(\b{j}_2) d\b{j}_1 d\b{j}_2\\
    &= \int (\b{j}_1^T \b{j}_2)^2 p(\b{j}_1|K_1) p(\b{j}_2|K_2) d\b{j}_1 d\b{j}_2 dK_1 dK_2\\
    &= \int S(O; K_1, K_2)^2 p(O|K_1, K_2) p(K_1) p(K_2) dO dK_1 dK_2\\
    &= \int S(O; K_1, K_2)^2 Hypergeom(N, K_1, K_2, O) p(K_1) p(K_2) dO dK_1 dK_2,
\end{align}
where 
\begin{align}
    S(O; K_1, K_2) &= O j_+^2 + (K_1 + K_2 - 2O) j_- j_+ + (N - K_1 - K_2 + O) j_-^2\\
    &= O(f_+^2 + f_-^2 - 2 f_+ f_-) + (K_1 + K_2)(f_+ f_- - f_-^2)+Nf_-\\
    &:= \alpha O + \beta
\end{align}
with $j_+ = 1 - \bar{J}$ and $j_- = 0 - \bar{J}$ in the presence of inhibition.
Now
\begin{equation}
    S(O; K_1, K_2)^2 = \alpha^2 O^2 + 2O \alpha \beta + \beta^2.
\end{equation}
We now note that
\begin{equation}
    \int O Hypergeom(N, K_1, K_2, O) dO = K_1 K_2 / N
\end{equation}
and
\begin{equation}
    \int O^2 Hypergeom(N, K_1, K_2, O) dO = 
    N^{-2} \left ( (K_1 K_2)^2 + \frac{K_1 K_2 (N-K_1) (N-K_2)}{N-1} \right ) := \gamma.
\end{equation}
This leads to
\begin{align}
    \braket{S_{ij}^2}
    &= \int (\alpha^2 \gamma + 2\alpha \beta K_1 K_2/N + \beta^2 ) p(K_1) p(K_2) dK_1 dK_2.
\end{align}
We now treat these terms one at a time, noting that $\alpha = (f_+^2 + f_-^2 - 2 f_+ f_-)$ does not depend on $K$ and defining $\beta = (K_1 + K_2)(f_+ f_- - f_-^2)+Nf_- := (K_1+K_2)\delta + \epsilon$.
\begin{align}
    I_1 &:= \int (\alpha^2 \gamma) p(K_1) p(K_2) dK_1 dK_2\\
    &= \alpha^2 \int \gamma p(K_1) p(K_2) dK_1 dK_2\\
    &= \alpha^2 N^{-2} \left ( \braket{K^2}^2 + (N-1)^{-1} (\braket{K}^2 N^2 + \braket{K^2}^2 - 2 N \braket{K} \braket{K^2}) \right )
\end{align}

\begin{align}
    I_2 &:= \int (2\alpha \beta K_1 K_2/N) p(K_1) p(K_2) dK_1 dK_2\\
    &= 2\alpha/N \int (((K_1+K_2)\delta + \epsilon) K_1 K_2) p(K_1) p(K_2) dK_1 dK_2\\
    &=2\alpha/N \left ( \epsilon \braket{K}^2 + 2 \delta \braket{K}\braket{K^2} \right )
\end{align}

\begin{align}
    I_3 &:= \int \beta^2 p(K_1) p(K_2) dK_1 dK_2\\
    &= \int ((K_1+K_2)\delta + \epsilon)^2 p(K_1) p(K_2) dK_1 dK_2\\
    &= \epsilon^2 + 4 \epsilon \delta \braket{K} + 2 \delta (\braket{K}^2 + \braket{K^2}).
\end{align}

\end{document}
